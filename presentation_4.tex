%%%%%%%%%%%%%%%%%%%%%%%%%%%%%%%%%%%%%%%%%
% Beamer Presentation
% LaTeX Template
% Version 1.0 (10/11/12)
%
% This template has been downloaded from:
% http://www.LaTeXTemplates.com
%
% License:
% CC BY-NC-SA 3.0 (http://creativecommons.org/licenses/by-nc-sa/3.0/)
%
%%%%%%%%%%%%%%%%%%%%%%%%%%%%%%%%%%%%%%%%%

%----------------------------------------------------------------------------------------
%	PACKAGES AND THEMES
%----------------------------------------------------------------------------------------

\documentclass{beamer}

\mode<presentation> {

% The Beamer class comes with a number of default slide themes
% which change the colors and layouts of slides. Below this is a list
% of all the themes, uncomment each in turn to see what they look like.

%\usetheme{default}
%\usetheme{AnnArbor}
%\usetheme{Antibes}
%\usetheme{Bergen}
%\usetheme{Berkeley}
%\usetheme{Berlin}
%\usetheme{Boadilla}
%\usetheme{CambridgeUS}
%\usetheme{Copenhagen}
%\usetheme{Darmstadt}
%\usetheme{Dresden}
%\usetheme{Frankfurt}
%\usetheme{Goettingen}
%\usetheme{Hannover}
%\usetheme{Ilmenau}
%\usetheme{JuanLesPins}
%\usetheme{Luebeck}
\usetheme{Madrid}
%\usetheme{Malmoe}
%\usetheme{Marburg}
%\usetheme{Montpellier}
%\usetheme{PaloAlto}
%\usetheme{Pittsburgh}
%\usetheme{Rochester}
%\usetheme{Singapore}
%\usetheme{Szeged}
%\usetheme{Warsaw}

% As well as themes, the Beamer class has a number of color themes
% for any slide theme. Uncomment each of these in turn to see how it
% changes the colors of your current slide theme.

%\usecolortheme{albatross}
%\usecolortheme{beaver}
%\usecolortheme{beetle}
%\usecolortheme{crane}
%\usecolortheme{dolphin}
%\usecolortheme{dove}
%\usecolortheme{fly}
%\usecolortheme{lily}
%\usecolortheme{orchid}
%\usecolortheme{rose}
%\usecolortheme{seagull}
%\usecolortheme{seahorse}
%\usecolortheme{whale}
%\usecolortheme{wolverine}

%\setbeamertemplate{footline} % To remove the footer line in all slides uncomment this line
%\setbeamertemplate{footline}[page number] % To replace the footer line in all slides with a simple slide count uncomment this line

%\setbeamertemplate{navigation symbols}{} % To remove the navigation symbols from the bottom of all slides uncomment this line
}

\usepackage{graphicx} % Allows including images
\usepackage{booktabs} % Allows the use of \toprule, \midrule and \bottomrule in tables

%----------------------------------------------------------------------------------------
%	TITLE PAGE
%----------------------------------------------------------------------------------------

\title{Chapter Four: Synthesis of Array Antennas} % The short title appears at the bottom of every slide, the full title is only on the title page
\author{By:Molla Belete} % Your name
\institute[] % Your institution as it will appear on the bottom of every slide, may be shorthand to save space
{University of Debre Tabor \\ % Your institution for the title page
\medskip
\textit{mole.ece2014@gmail.com} % Your email address
}
\date{\today} % Date, can be changed to a custom date

%\begin{document}
\begin{document}
	\begin{frame}
		\titlepage % Print the title page as the first slide
	\end{frame}
	
	\begin{frame}
		\frametitle{Outlines} 
		\tableofcontents 
	\end{frame}
	
	\begin{frame}
		\section{Introduction}
		\frametitle{Introduction}
		\begin{itemize}
			\item The radiation patterns of single-element antennas are relatively wide,
			i.e., they have relatively low directivity (gain).
			\item In long distance communications, antennas with high directivity are required
			\begin{itemize}
				\item Possible to construct by enlarging the dimensions of the radiating aperture
				(maximum size much larger than $\lambda$)
				\item To increase the electrical size of an antenna is to construct it as an assembly of radiating elements in a proper electrical and geometrical configuration called \color{blue}antenna array\color{black}
			\end{itemize}
			\item Usually, the array elements are identical. This is not necessary but it is practical and simpler for design and fabrication. 
			\item The individual elements may be of any type (wire dipoles, loops, apertures, etc.) 
			\item The total field of an array is a vector superposition of the
			fields radiated by the individual elements. 
		\end{itemize}
	\end{frame}
	
	\begin{frame}
		\begin{itemize}
			\item To provide very directive pattern, it is necessary that the partial fields (generated by the individual elements) interfere \color{blue}constructively \color{black}in the desired direction and interfere \color{blue}destructively \color{black}
			in the remaining space.
			\item There are five basic methods to control the overall antenna pattern:
			\begin{enumerate}
				\item The geometrical configuration of the overall array (linear, circular,
				spherical, rectangular, etc.),
				\item The relative placement of the elements,
				\item The excitation amplitude of the individual elements,
				\item The excitation phase of each element,
				\item The individual pattern of each element
			\end{enumerate}
		\end{itemize}
	\end{frame}
	\begin{frame}
		\section{Two-Element Array}
		\frametitle{Two-Element Array}
		\begin{itemize}
			\item Assume we have an array of two infinitesimal horizontal dipoles positioned
			along the z -axis
			\begin{description}
				\item[] Where the first element is excited by current $ I_1 = I_0 e^ {−j\beta/2} $ and the second by $ I_2 = I_0 e^{+j\beta/2} $
			\end{description}
			\begin{figure}
				\centering
				\includegraphics[width=1\linewidth]{2ElArray}
			\end{figure}
		\end{itemize}
	\end{frame}
	\begin{frame}
		\begin{itemize}
			\item The total field radiated by the two
			elements is
			\begin{align}
			E_t &= E_1 + E_2 \nonumber\\
			&= \mathbf{a_\theta}j\eta\dfrac{kI_0 l}{4\pi }\left(\dfrac{e^{-j(kr_1 - \beta/2)}}{r_1}cos\theta_1 + \dfrac{e^{-j(kr_1 - \beta/2)}}{r_2}cos\theta_2 \right)
			\end{align}
			\begin{description}
				\item At far-field
				\begin{description}
					\item[$ \theta_1 \simeq \theta_2 = \theta $] 
					\item[$ r_1 \simeq r - \dfrac{d}{2}cos{\theta} $] for phase variation
					\item[$ r_2 \simeq r - \dfrac{d}{2}cos{\theta} $] for phase variation
					\item[$ \theta_1 \simeq \theta_2 = \theta $] for amplitude variation
				\end{description}
				\begin{align}
				E_t&= \mathbf{a_\theta}j\eta\dfrac{kI_0 l}{4\pi r}cos\theta\left(e^{+j(kdcos\theta + \beta)/2} + e^{-j(kdcos\theta + \beta)/2} \right)\nonumber
				\end{align}
				\begin{center}
					OR
				\end{center}
			\end{description}
		\end{itemize}
	\end{frame}
	\begin{frame}
		\begin{equation} 
		E_t= \underbrace{\mathbf{a_\theta}j\eta\dfrac{kI_0 l}{4\pi r}cos\theta}_\text{Element Factor} \underbrace{2cos\left[\dfrac{1}{2}(kdcos\theta + \beta)  \right]}_\text{Array Factor}
		\end{equation}
		\begin{itemize}
			\item Thus the total field of the array is equal to the product of the field created
			by a single element located at the origin and the array factor, AF :
			\begin{equation}
			AF = 2cos\left[\dfrac{1}{2}(kdcos\theta + \beta)\right]
			\end{equation}
			\begin{description}
				\item[] In normalized form
				\begin{equation}
				AF = cos\left[\dfrac{1}{2}(kdcos\theta + \beta)\right]
				\end{equation}
			\end{description}
			\item The far-zone field of a uniform two-element array of identical elements is the product of the field of a single element and the array factor of that array
			\boxedeq{}{E_t = [E(single\hskip0.01in element\hskip0.01in at\hskip0.01in reference\hskip0.01in point)] \times [array\hskip0.01in factor]}
		\end{itemize}
	\end{frame}
	\begin{frame}
		\begin{itemize}
			\item Pattern multiplication rule valid for arrays of identical elements. 
			\begin{itemize}
				\item This rule holds for any array consisting of decoupled identical elements, where the excitation magnitudes, the phase shift between the elements and the displacement between them are not necessarily the same. 
			\end{itemize}
			\item The total pattern can be controlled via
			the single-element pattern, or via the AF. 
			\item Generally the AF depends on
			the:
			\begin{enumerate}
				\item The number of elements,
				\item The mutual placement,
				\item The relative excitation magnitudes and phases.
			\end{enumerate}
		\end{itemize}
	\end{frame}
	\begin{frame}
		\section{N-Element Linear Array with Uniform Amplitude and Spacing}
		\frametitle{N-Element Linear Array with Uniform Amplitude and Spacing}
		\begin{itemize}
			\item Assume that each succeeding element has a $\beta$ progressive phase lead current excitation relative to the preceding one
			\item An array of identical elements with identical magnitudes and with a progressive phase is called a uniform	array. 
			\item The AF of the uniform array can be obtained by considering the individual elements as point (isotropic) sources. 
			\item The total field pattern can be obtained by simply multiplying the AF by the field pattern of the
			individual element (provided the elements are not coupled)
		\end{itemize}
	\end{frame}
	
	\begin{frame}
		\begin{itemize}
			\item The AF of an N -element linear array of isotropic sources is
			\begin{equation}
			AF = 1+e^{j(kdcos\theta + \beta)}+e^{2j(kdcos\theta +  \beta)}+\ldots+e^{j(N-1)(kdcos\theta + \beta)}
			\end{equation}
			\boxedeq{}{AF &= \sum_{n = 1}^{N} e^{j(n-1)\psi}\\
				Where\hskip0.1in &\psi = kdcos\theta +  \beta\nonumber}
			\item More convenient for pattern analysis:
			\boxedeq{}{AF & = e^{j[(N-1)/2]\psi}\left[ \dfrac{sin\left(\dfrac{N}{2}\psi\right)}{sin\left(\dfrac{\psi}{2}\right)}\right]}
			\item The phaser factor $e^{j[(N-1)/2]\psi}$
			\begin{itemize}
				\item Not important unless the array output signal is further combined with the output
				signal of another antenna
				\item It represents the phase shift of the array’s phase center relative to the origin
			\end{itemize}
		\end{itemize}
	\end{frame}
	\begin{frame}
		\begin{itemize}
			\item Neglecting the phase factor gives
			\begin{align}
			AF & = \left[ \dfrac{sin\left(\dfrac{N}{2}\psi\right)}{sin\left(\dfrac{\psi}{2}\right)}\right]
			\end{align}
			
			\item Normalizing the array factor we obtain
			\boxedeq{}{AF_n & = \dfrac{1}{N}\left[ \dfrac{sin\left(\dfrac{N}{2}\psi\right)}{sin\left(\dfrac{\psi}{2}\right)}\right]}
			\item For small values of $ \psi $, it reduces to
			\boxedeq{}{
				AF_n & \simeq \dfrac{1}{N}\left[ \dfrac{sin\left(\dfrac{N}{2}\psi\right)}{\dfrac{\psi}{2}}\right]
			}
		\end{itemize}
	\end{frame}
	\begin{frame}
		\begin{description}
			\item[Null:] To find the nulls of the AF, equation (10) is set equal to zero
			\boxedeq{}{\theta_n& = cos^{-1}\left[\dfrac{\lambda}{2\pi d}\left(-\beta \pm \dfrac{2n}{N}\pi\right)\right],\\ 
				&n = 1, 2, 3, \ldots (n \neq 0, N, 2N, 3N, \ldots)\nonumber}
			\item[Maxima:] They are studied in order to determine the maximum directivity, the HPBWs,	the direction of maximum radiation
			\item[] The maximum value of (10) occur	when
			\begin{equation}
			\dfrac{\psi}{2} = \dfrac{1}{2}(kdcos\theta_m +\beta) = \pm m\pi \nonumber
			\end{equation}
			\boxedeq{}{\theta_m = cos^{-1}\left[\dfrac{\lambda}{2\pi d}\left(-\beta \pm 2m\pi\right)\right], m = 0, 1, 2, \ldots}
		\end{description}
	\end{frame}
	\begin{frame}
		\begin{description}
			\item[HPBW] Calculated by setting the value of $ AF_n $ equal to $1/\sqrt{2}$
			\boxedeq{}{\theta _h& = cos^{-1}\left[\dfrac{\lambda}{2\pi d}\left(-\beta \pm \dfrac{2.782}{N}\right)\right]}
			\item[] For a symmetrical pattern around $ \theta_m $ (the angle at which maximum radiation occurs), the HPBW is calculated as
			\boxedeq{}{\Theta _h& =2|\theta_h - \theta_m|}
			\item[Maxima of Minor Lobes:] They are the maxima of $ AF_n $ , where $ AF_n < 1 $. 
			\item[] They occur approximately
			where the numerator attains a maximum and the AF is beyond its first null:
			\begin{equation}
			sin\left(\dfrac{N}{2}\psi\right) = \pm 1\Rightarrow \dfrac{N}{2}(kdcos\theta_s +\beta) = \pm (2s+1)\frac{\pi}{2}\nonumber
			\end{equation}
		\end{description}
	\end{frame}
	
	\begin{frame}
		\begin{description}
			\item[]\boxedeq{}{\theta_s& = cos^{-1}\left[\dfrac{\lambda}{2\pi d}\left(-\beta \pm \dfrac{2s+1}{N}\pi\right)\right],\\
				&s = 1, 2, 3, \ldots\nonumber }
			\item[] It can be also written as
			\begin{align}
			\theta_s& = \dfrac{\pi}{2} - sin^{-1}\left[\dfrac{\lambda}{2\pi d}\left(-\beta \pm \dfrac{2s+1}{N}\pi\right)\right],\\
			&s = 1, 2, 3, \ldots\nonumber
			\end{align}
			\item[] For large values of $ d (d \ll \lambda) $, it reduces to
			\begin{align}
			\theta_s& = \dfrac{\pi}{2} - \left[\dfrac{\lambda}{2\pi d}\left(-\beta \pm \dfrac{2s+1}{N}\pi\right)\right],\\
			&s = 1, 2, 3, \ldots\nonumber
			\end{align}
			\item[] The maximum of the first minor lobe occurs when $ s = 1 $
			\begin{equation}
			AF_N = 0.212 = -13.46dB\nonumber
			\end{equation}
			\item[] The maximum of the first minor lobe of the array factor is	13.46 dB down from the maximum at the major lobe.
		\end{description}
	\end{frame}
	
	\begin{frame}
		\subsection{Broadside Array}
		\frametitle{Broadside Array}
		\begin{itemize}
			\item An array, which has maximum radiation at $\theta = 90^\circ $ (normal to the axis of the array).
			\item The maximum of the AF occurs when $\psi = 0$. At $\theta = 90^\circ$
			\boxedeq{}{\beta = 0}
			\begin{block}{}
				The uniform linear array has its maximum radiation at $ \theta = 90^\circ $ , if all array
				elements have their excitation with the same phase ($ \beta = 0 $)
			\end{block}
			\item To ensure that there are no maxima in the other directions (called grating
			lobes), the separation between the elements should not be equal to multiples
			of a wavelength:
			\begin{equation}
			d \neq n\lambda,\hskip0.1in n = 1, 2, 3, \ldots
			\end{equation}
		\end{itemize}
	\end{frame}
	\begin{frame}
		\begin{itemize}
			\item For a uniform array with $ \beta = 0 $ and $ d = n\lambda $, in addition to having the maxima of the array factor directed broadside ($ \theta = 90^\circ $ )
			to the axis of the array, 
			\begin{itemize}
				\item There are additional maxima directed along the axis ($\theta = 0^\circ$ , $180^\circ$ ) of the array (end-fire radiation).
			\end{itemize}
			\item One of the objectives in many designs is to avoid multiple maxima (grating
			lobes)
			\item To avoid any grating lobe, the largest
			spacing between the elements should be less than one wavelength ($ d_{max} < \lambda $)
			\begin{figure}
				\centering
				\includegraphics[width=0.8\linewidth]{Broadarr}
				\caption{}
				\label{fig:broadarr}
			\end{figure}
			
		\end{itemize}
	\end{frame}
	\begin{frame}
		\subsection{Ordinary End-Fire Array}
		\frametitle{Ordinary End-Fire Array}
		\begin{itemize}
			\item An array which has its maximum radiation along the	axis of the array ($ \theta = 0^\circ , 180^\circ $ ). 
			\item It may be required that the array radiates only in one direction- either $ \theta = 0^\circ or 180^\circ$ .
			\item[] For an AF maximum at $\theta = 0^\circ$ ,
			\boxedeq{}{\beta = kd}
			\item If the element separation is multiple of a wavelength, $d = n\lambda$, then in ad-
			dition to the end-fire maxima there also exist maxima in the broadside
			directions. 
			\item As with the broadside array, in order to avoid grating lobes, the
			maximum spacing between the element should be less than $\lambda$:
			\begin{equation}
			d_{max} < \lambda
			\end{equation}
		\end{itemize}
	\end{frame}
	\begin{frame}
		\begin{figure}
			\centering
			\includegraphics[width=1\linewidth]{endfirearr}
			\label{fig:endfirearr}
		\end{figure}
	\end{frame}
	\begin{frame}
		\subsection{Phased (Scanning) Arrays}
		\frametitle{Phased (Scanning) Arrays}
		\begin{itemize}
			\item It is logical to assume that the maximum radiation can be oriented in any direction to form a scanning array. 
			\item Let the maximum radiation of the array is required to be oriented at angle $ \theta_0 $ $ (0^\circ \leq \theta \leq 180^\circ ) $. To accomplish this, the phase excitation
			$\beta$ between the elements must be adjusted so that 
			\begin{equation}
			\psi = kdcos\theta +\beta|_{\theta = \theta_0} = kdcos\theta_0 +\beta = 0\nonumber 
			\end{equation}
			$\Rightarrow$ \boxedeq{}{\beta = -kdcos\theta_0}	
		\end{itemize}
	\end{frame}
	\begin{frame}
		\subsection{Hansen-Woodyard End-Fire Array}
		\frametitle{Hansen-Woodyard End-Fire Array}
		\begin{itemize}
			\item To enhance the directivity of an end-fire
			array, Hansen and Woodyard proposed that the phase shift of an ordinary
			end-fire $ \beta = \pm kd $ be increased for closely spaced elements of a very long array as
			\begin{align}
			\beta & = -\left(kd+\dfrac{2.94}{N}\right)\simeq-\left(kd+\dfrac{\pi}{N}\right), for\hskip0.01in max\hskip0.01in in\hskip0.05in \theta = 0^\circ\\
			\beta & = +\left(kd+\dfrac{2.94}{N}\right)\simeq+\left(kd+\dfrac{\pi}{N}\right), for\hskip0.01in max\hskip0.01in in\hskip0.05in \theta = 0^\circ
			\end{align}
			\item[] are known as the Hansen-Woodyard conditions
			for end-fire radiation. They follow from a procedure for maximizing the directivity.
		\end{itemize}
	\end{frame}
	\begin{frame}
		\section{Directivity of a Linear Array}
		\frametitle{Directivity of a Linear Array}
		\color{blue}Directivity of Broadside Array\color{black}
		\begin{align}
		U(\theta)& = |AF_N|^2 = \left[ \dfrac{sin\left(\dfrac{N}{2}kdcos\theta\right)}{\dfrac{N}{2}kdcos\theta}\right]^2 = \left[\dfrac{sinZ}{Z}\right]^2\\
		U_{max}& = U(\theta = \frac{\pi}{2}) = 1
		\end{align}
		the radiation intensity averaged over all directions is
		\begin{align}
		P_{av}& = \frac{1}{2}\int_{0}^{\pi}\left[ \dfrac{sin\left(\dfrac{N}{2}kdcos\theta\right)}{\dfrac{N}{2}kdcos\theta}\right]^2 sin\theta d\theta\\
		&= \dfrac{1}{Nkd}\int_{-Nkd/2}^{Nkd/2}\left[\dfrac{sinZ}{Z}\right]^2dZ = \dfrac{\pi}{Nkd}
		\end{align}
	\end{frame}
	\begin{frame}
		\begin{description}
			\item[]The directivity becomes
			\boxedeq{}{D = 2N\left(\dfrac{d}{\lambda}\right)}
			The length of the array $ L = (N - 1)d $
			\begin{equation}
			D_0 = 2\left(1+\frac{L}{d}\right)\left(\frac{d}{\lambda}\right)
			\end{equation}
			\item[]For large arrays ($ L\gg d $)
			\begin{equation}
			D_0 \simeq 2\left(\frac{L}{\lambda}\right)
			\end{equation}
		\end{description}
		\color{blue}Directivity of Ordinary End-Fire Array\color{black}
		\begin{description}
			\item[] In a similar way the directivity of an end-fire array
			becomes
			\boxedeq{}{D =  \dfrac{U_{max}}{U_av}= 4N\left(\dfrac{d}{\lambda}\right)}
		\end{description}
	\end{frame}
	\begin{frame}
		\begin{description}
			\item[] The length of the array $ L = (N - 1)d $
			\begin{equation}
			D_0 = 4\left(1+\frac{L}{d}\right)\left(\frac{d}{\lambda}\right)
			\end{equation}
			\item[]For large arrays ($ L\gg d $)
			\begin{equation}
			D_0 \simeq 4\left(\frac{L}{\lambda}\right)
			\end{equation}
		\end{description}
		\color{blue}Directivity of Hansen-Woodyard Array\color{black}
		\begin{description}
			\item[] The directivity of Hansen-Woodyard array is
			\boxedeq{}{D =  1.805\left[4N\left(\dfrac{d}{\lambda}\right)\right]}
			\item[] The length of the array $ L = (N - 1)d $
			\begin{equation}
			D_0 =1.805\left[ 4\left(1+\frac{L}{d}\right)\left(\frac{d}{\lambda}\right)\right]
			\end{equation}
		\end{description}
	\end{frame}
	\begin{frame}
		\begin{description}
			\item[]For large arrays ($ L\gg d $)
			\begin{equation}
			D_0 \simeq 1.805\left[4\left(\frac{L}{\lambda}\right)\right]
			\end{equation}
			\item[Exercise] Given a linear uniform array of isotropic elements with $ N = 10 $, $ d =\lambda/4 $, find the directivity if:\\
			1. $ \beta = 0 $ (broadside)\\
			2. $ \beta = −kd $ (end-fire)\\
			3. $ \beta = −kd − \pi/N $ (Hansen-Woodyard)\\
			\item[[ans. a. 5 (=6.999 dB) b. 10 (10 dB) c. 17.89 (12.53 dB)]] 
		\end{description}
	\end{frame}
	
	\begin{frame}
		\Huge{\centerline{Thank You!!!}}
	\end{frame}
\end{document}
%\begin{frame}
%\titlepage % Print the title page as the first slide
%\end{frame}
%
%\begin{frame}
%\frametitle{Overview} % Table of contents slide, comment this block out to remove it
%\tableofcontents % Throughout your presentation, if you choose to use \section{} and \subsection{} commands, these will automatically be printed on this slide as an overview of your presentation
%\end{frame}
%
%%----------------------------------------------------------------------------------------
%%	PRESENTATION SLIDES
%%----------------------------------------------------------------------------------------
%
%%------------------------------------------------
%\section{First Section} % Sections can be created in order to organize your presentation into discrete blocks, all sections and subsections are automatically printed in the table of contents as an overview of the talk
%%------------------------------------------------
%
%\subsection{Subsection Example} % A subsection can be created just before a set of slides with a common theme to further break down your presentation into chunks
%
%\begin{frame}
%\frametitle{Paragraphs of Text}
%Sed iaculis dapibus gravida. Morbi sed tortor erat, nec interdum arcu. Sed id lorem lectus. Quisque viverra augue id sem ornare non aliquam nibh tristique. Aenean in ligula nisl. Nulla sed tellus ipsum. Donec vestibulum ligula non lorem vulputate fermentum accumsan neque mollis.\\~\\
%
%Sed diam enim, sagittis nec condimentum sit amet, ullamcorper sit amet libero. Aliquam vel dui orci, a porta odio. Nullam id suscipit ipsum. Aenean lobortis commodo sem, ut commodo leo gravida vitae. Pellentesque vehicula ante iaculis arcu pretium rutrum eget sit amet purus. Integer ornare nulla quis neque ultrices lobortis. Vestibulum ultrices tincidunt libero, quis commodo erat ullamcorper id.
%\end{frame}
%
%%------------------------------------------------
%
%\begin{frame}
%\frametitle{Bullet Points}
%\begin{itemize}
%\item Lorem ipsum dolor sit amet, consectetur adipiscing elit
%\item Aliquam blandit faucibus nisi, sit amet dapibus enim tempus eu
%\item Nulla commodo, erat quis gravida posuere, elit lacus lobortis est, quis porttitor odio mauris at libero
%\item Nam cursus est eget velit posuere pellentesque
%\item Vestibulum faucibus velit a augue condimentum quis convallis nulla gravida
%\end{itemize}
%\end{frame}
%
%%------------------------------------------------
%
%\begin{frame}
%\frametitle{Blocks of Highlighted Text}
%\begin{block}{Block 1}
%Lorem ipsum dolor sit amet, consectetur adipiscing elit. Integer lectus nisl, ultricies in feugiat rutrum, porttitor sit amet augue. Aliquam ut tortor mauris. Sed volutpat ante purus, quis accumsan dolor.
%\end{block}
%
%\begin{block}{Block 2}
%Pellentesque sed tellus purus. Class aptent taciti sociosqu ad litora torquent per conubia nostra, per inceptos himenaeos. Vestibulum quis magna at risus dictum tempor eu vitae velit.
%\end{block}
%
%\begin{block}{Block 3}
%Suspendisse tincidunt sagittis gravida. Curabitur condimentum, enim sed venenatis rutrum, ipsum neque consectetur orci, sed blandit justo nisi ac lacus.
%\end{block}
%\end{frame}
%
%%------------------------------------------------
%
%\begin{frame}
%\frametitle{Multiple Columns}
%\begin{columns}[c] % The "c" option specifies centered vertical alignment while the "t" option is used for top vertical alignment
%
%\column{.45\textwidth} % Left column and width
%\textbf{Heading}
%\begin{enumerate}
%\item Statement
%\item Explanation
%\item Example
%\end{enumerate}
%
%\column{.5\textwidth} % Right column and width
%Lorem ipsum dolor sit amet, consectetur adipiscing elit. Integer lectus nisl, ultricies in feugiat rutrum, porttitor sit amet augue. Aliquam ut tortor mauris. Sed volutpat ante purus, quis accumsan dolor.
%
%\end{columns}
%\end{frame}
%
%%------------------------------------------------
%\section{Second Section}
%%------------------------------------------------
%
%\begin{frame}
%\frametitle{Table}
%\begin{table}
%\begin{tabular}{l l l}
%\toprule
%\textbf{Treatments} & \textbf{Response 1} & \textbf{Response 2}\\
%\midrule
%Treatment 1 & 0.0003262 & 0.562 \\
%Treatment 2 & 0.0015681 & 0.910 \\
%Treatment 3 & 0.0009271 & 0.296 \\
%\bottomrule
%\end{tabular}
%\caption{Table caption}
%\end{table}
%\end{frame}
%
%%------------------------------------------------
%
%\begin{frame}
%\frametitle{Theorem}
%\begin{theorem}[Mass--energy equivalence]
%$E = mc^2$
%\end{theorem}
%\end{frame}
%
%%------------------------------------------------
%
%\begin{frame}[fragile] % Need to use the fragile option when verbatim is used in the slide
%\frametitle{Verbatim}
%\begin{example}[Theorem Slide Code]
%\begin{verbatim}
%\begin{frame}
%\frametitle{Theorem}
%\begin{theorem}[Mass--energy equivalence]
%$E = mc^2$
%\end{theorem}
%\end{frame}\end{verbatim}
%\end{example}
%\end{frame}
%
%%------------------------------------------------
%
%\begin{frame}
%\frametitle{Figure}
%Uncomment the code on this slide to include your own image from the same directory as the template .TeX file.
%%\begin{figure}
%%\includegraphics[width=0.8\linewidth]{test}
%%\end{figure}
%\end{frame}
%
%%------------------------------------------------
%
%\begin{frame}[fragile] % Need to use the fragile option when verbatim is used in the slide
%\frametitle{Citation}
%An example of the \verb|\cite| command to cite within the presentation:\\~
%
%This statement requires citation \cite{p1}.
%\end{frame}
%
%%------------------------------------------------
%
%\begin{frame}
%\frametitle{References}
%\footnotesize{
%\begin{thebibliography}{99} % Beamer does not support BibTeX so references must be inserted manually as below
%\bibitem[Smith, 2012]{p1} John Smith (2012)
%\newblock Title of the publication
%\newblock \emph{Journal Name} 12(3), 45 -- 678.
%\end{thebibliography}
%}
%\end{frame}
%
%%------------------------------------------------
%
%\begin{frame}
%\Huge{\centerline{The End}}
%\end{frame}
%
%%----------------------------------------------------------------------------------------
%
%\end{document} 